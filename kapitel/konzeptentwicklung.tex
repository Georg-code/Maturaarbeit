
\chapter{Konzeptentwicklung }
\label{chap:konzeptentwicklung}


Es wird ein Boot konstruiert und gebaut, welches über die Mittel der Autonomen Navigation verfügt. Dabei steht vorallem die Konstruktion des Bootes im Vordergrund.

\section{Anforderungsanalyse}
Das Boot muss über 
\section{Ideenfindung und Konzeptauswahl}
Um eine Richtung und Einsatzbereich zu finden, wird als erstes die Literatur in diesem Bereich 
\section{Erläuterung des ausgewählten Sailflap-Designs}


\section{Energie}
Da das Segelboot eine "inteligente" Komponente haben sollte, braucht dieses elektrischen Strom um die Komponenten mit Energie zu versorgen. Da das Segelboot vollkommen Autonom funktionieren soll ist es sinnvoll diese auf dem Schiff zu generieren. In frage kommen dabei drei Systeme. Zum einen die Generierung von Elektrischem Strom mithilfe von Wasserkraft.
Da sich das Boot im Wasser bewegt, wäre es möglich eine kleine Turbine im Wasser zu Platzieren. Da sich das Wasser relativ zum Boot bewegt sind die gleichen prinzipien wie bei Generierung von Wasserkraft anzuwenden. \\
Diese Methode hat jedoch auch Nachteile. Zum ist das Segelboot auf geringe Geschwindigkeiten ausgelegt. Da die Leistung einer Turbine in einem Fluid kubisch ansteigt wäre es nur möglich geringe Mengen zu Produzieren. Zudem würde dies das Boot noch weiter ausbremsen. Ähnliche Probleme hat ebenfalls die Generierung von elektrischem Strom mithilfe von Windkraft. Zudem wäre die Mechanik hinter einer solchen Anlage sehr komplex da sich diese immer zum Wind drehen müsste, die Balance des Schiffes schaden würde und wenig Ertrag liefern würde.
Daher bleibt noch die Photovoltaik. Die Energieerzeugung mittels Solaranlagen ist auf Segelbooten äusserst bleibt, da es keinen zusätzlichen Widerstand bringt. Der gravierende Nachteil ist jedoch, dass der Sonnenzyklus und das Wetter einen entscheidenden Einfluss auf die Energiegewinnung haben. An einem sonnigen Tag sind diese sehr ergiebig, wobei in der Nacht kein Strom generiert werden kann.



Da das Segelboot vollkommen autonom funktionieren soll, muss die Energieversorgung von Anfang an beachtet werden. Daher wurde ein Konzept entwickelt um das Segelboot mit Solarenergie zu versorgen. Dies beinhaltet einen Akku welcher über genügend Kapazität verfügen muss. \\
Da sich sehr schnell herausstellt, dass im Handel verfügbare Powerbanks für diese Anwendung unbrauchbar sind. Dies liegt daran, dass kaum eine Powerbank "pass trough loading" unterstützt. Diese Technonologie kennt man in der Regel nur von einem UPS (Uninteruppted Power Supply). Ein solches ist jedoch in der Grössenskala sehr schwer zu erwerben. 


\section{Windrichtungssensor}








