
\chapter{Konzeptentwicklung }
\label{chap:konzeptentwicklung}
\section{Zielsetzung}
Das Ziel dieser Arbeit ist es, ein autonomes Segelboot zu entwerfen, zu konstruieren und zu bauen. Dabei soll günstiges und einfach zu verarbeitendes Konstruktonsmaterial verwendet werden, das in Baumärkten erworben werden kann, mit Werkzeugen gearbeitet werden, die in einer gut bestückten Handwerkerwerkstatt vorhanden sind, die Navigation und Steuerung selbst entworfen werden und die dafür notwendigen Programme selbst entwickelt werden. 

?????? Dazu soll mittels eines CAD-Programmes (CAD = Computer Aided Design) computergestützt ein Segelboot entworfen werden, um anschliessend nach den mit dem CAD-Programm erstellten Plänen gebaut zu werden. Das Segelboot soll mit allen zur autonomen Navigation und Steuerung erforderlichen Geräten, Rechnern und Sensoren sowie der zu deren Betrieb erforderlichen autonomen Energieversorgung ausgerüstet werden. Schliesslich soll das Segelboot mit Programmen ausgestattet werden, die es ihm erlauben, bei allen möglichen Windrichtungen den vordefinierten Zielpunkt anzusteuern und selbst bei einer Veränderung der Windrichtung selbständig zu erreichen.

\section{Methodische Vorgehen}

In einem ersten Schritt wird ein Überblick über die bekannten autonomen Segelboote und Segelbootprojekte gewonnen. Im obigen Kapitel \ref{chap:literaturübersicht} werden die relevanten Arbeiten aufgeführt. Anschliessend wird geklärt, ob ein autonomes Segelboot in der Schweiz überhaupt betrieben werden darf (Kapitel 3.3). Dann werden die Anforderungen an ein autonomes Segelboot, das den Zielsetzung dieser Arbeit entspricht, analysiert (Kapitel 3.4). Im folgenden Schritt werden Überlegungen zum Design angestellt. Dabei werden die möglichen Segelbootstypen anhand der Hauptelement Rumpfzahl, Segelart und Kielart und deren Bedeutung für das Projekt diskutiert. Anschliessend wird der Designentscheid und der Materialentscheid getroffen und begründet. In nächsten Schritt wird der Konstruktionsprozess und die Konstruktion beschrieben. Danach erfolgt die Beschäftigung mit der Elektronik, also dem Rechner, den Sensoren und den Motoren,  



Konzeption, der Entwurf, die Konstruktion und der Bau des Bootskörpers sowie des Segels beschrieben. In einem zweiten Teil wird die

  Im Abschnitt über die technische/elektrische Ausrüstung des Bootes, insbesondere der verwendet Rechner und die Überlegungen zu dessen Wahl, die eingesetzten Sensoren, die Stromversorgungslösung und die Lengsteuerung diskutiert.

 In einem nächsten Teil werden nach einer knappen Einführung in die Grundlagen der Physik des Segelns, die Überlegungen zur Architektur der Naviationssoftware, die Navigation und die Wegfindung diskutiert

 In einem ….. Teil wird der Bau der Bestandteile des Bootskörpers und des Segels und anschliessende Zusammenbau und die Montage beschrieben 

\section{Gesetzliche Rahmenbedingungen}

Das projektierte autonome Segelboot ist ein Segelschiff im Sinne der Binnenschifffartsverordnung (siehe Kapital 1.2 oben) und fällt damit grundsätzlich unter die einschlägigen gesetzlichen Zulassungs-, Betriebs- und Verkehrsbestimmungen. Da es aber kürzer als 2.5 m ist, ist es von der Kennzeichnungspflicht mit einem behördlich zugeteilten Kennzeichen ausgenommen (Artikel 16 Abs. 2 Buchstabe b der Binnenschifffartsverordnung). Es benötigt damit gemäss Art. 92 der Binnenschiffartsverordnung auch keinen Schiffsausweis, womit auch eine behördliche Zulassungsprüfung entfällt. Das Boot muss aber einen Schiffsnamen tragen, der sich sowohl aus Buchstaben als auch aus Zahlen zusammensetzen kann. Ausserdem muss es mit dem Namen und der Adresse des Eigentümers versehen sein (Art. 16 Abs. 3 Binnenschifffartsverordnung). 

Da Art. 42 der Binnenschifffahrtsverordnung schreibt vor, dass Schiffe, die kürzer als 2,5 m sind, nur in der inneren Uferzone (150 m) oder im Abstand von höchstens 150 m um sie begleitende Schiffe herum verkehren dürfen. Autonome Segelboote in der Grössenordnung dieses Projektes können in der Schweiz damit nicht einfach sich selbst überlassen werden, sondern müssen im Einsatz aus der Distanz von einem andern, bemannten Schiff  begleitet und beobachtet werden.  

Gemäss Art. 16 Abs. 1 des Bundesgesetzes über die Binnenschifffahrt (BSG) vom 3. Oktober 1975 muss jedes Schiff einen verantwortlichen Führer haben. Da sich autonome Segelboote definitionsgemäss gerade dadurch auszeichnen, dass sie keinen Schiffsführer haben, scheint diese Bestimmung einem Einsatz von autonomen Segelboote in der Schweiz entgegen zustehen. Nach Abs. 2 dieser Bestimmung gilt als Schiffsführer, wer die tatsächliche Befehlsgewalt innehat. Entscheidend dabei ist, dass nicht verlangt wird, dass die Befehlsgewalt tatsächlich dauernd ausgeübt wird, sondern dass es genügt, wenn diese ausgeübt werden kann. Hierbei muss es genügen, wenn eine natürliche Person jederzeit mittels Funk Befehle an das autonomes Schiff senden kann und sich das Boot jederzeit in Sichtweite der Person befindet. Letztes ist auf Grund der Vorgaben von Art. 42 der Binnenschifffahrtsverordnung sowieso erforderlich und bedeutet deshalb keine zusätzliche Einschränkung.

Unter diesen Parametern ist der Betrieb eines autonomen Segelbootes aber ohne einen Schiffsfüherausweis möglich, solange dessen Segelfläche nicht mehr als 15 Quadratmeter beträgt (Art. 78 Abs. 1 Buchstabe b Binnenschifffahrtsverordnung). Es besteht auch keine Haftpflichtversicherungspflicht (Art. 153 Abs. 1 Binnenschifffahrtsverordnung). 

Bei Nacht oder bei unsichtigem Wetter muss das Segelboot mit a. getrennten Seitenlichtern und einem Hecklicht; b. einem Kombinations-Seitenlicht und einem Hecklicht; c einem Dreifarben-Topplicht; oder
d. einem weisses Rundumlicht ausgestattet werden (Art. 25 Abs. 2 Binnenschifffahrtsverordnung)

Ein autonomes Segelboot mit einer Länge von weniger als 2,5 m und einer Segelfläche von maximal 15 Quadratmetern kann auf Schweizer Gewässern also verkehren, sofern es von einem anderen Boot begleitet wird, welches maximal 150 m entfernt ist, und sofern eine natürliche Person auf diesem Begleitboot Befehlsgewalt über das autonome Segelboot ausüben kann, zum Beispiel indem diesem per Funk Befehle übermittelt werden oder indem es physich behändigt wird.

\section{Anforderungsanalyse}
Ausgehend von der Zielsetzung werden in einem ersten Schritt die Anforderungen an das autonome Segelboot analysiert und die Kriterien festgehalten, Das Boot muss über
\begin{itemize}
    \item stabil schwimmen
    \item den Wind zum Antrieb mit Segeln nutzen (keine Windturbine)
    \item günstig sein
    \item steuerbar sein
    \item an Land mit normalen Motorfahrzeugen an ein Gewässer transportiert werden können
    \item zwischen 2 und 2,5 m lang sein
    \item selbständig den Weg ins Ziel finden und ansteuern können
    \item selbständig mehrere, vorgegebene Ziel in der vorgegebenen Reihenfolge ansteuern können
    \item über eine autarke Energieversorgung verfügen und über eine Energiereserve von 72 h verfügen 
    \item muss auf Schweizer Gewässern betrieben werden können 
    \item Segelfläche von weniger als 15m2
    \item Beleuchtung
    \item Zieleingabe kann drahtlos erfolgen (Wlan)
    \item darf eine Länge von 2.5 m nicht überschreiten (wegen Zulassung)
    \item Muss über einen Namen verfügen und mit dem Namen und der Adresse des Eigentümers versehen sein 
    \item bei starkem Wellengang und starkem Wind nicht kentern
    \item darf nicht aus Metall aufbegaut sein, da keine Metallbearbeitungsmaschinen vorhanden sind
    \item sich in einem See bewegen können, ohne auf Grund zu laufen, wenn dem Boot eine Karte des Sees eingespeist wird ????
\end{itemize}

Kein Fokus auf Geschwindigkeit
Kein Fokus auf der Ermittlung des schnellsten Weges, um alle vorgegebenen Zielpunkte abzufahren (Problem des Handlungsreisenden)
Keine Erkennung von Hindernissen wie andere Boote etc.
 

\section{Bestätigung des Entscheids für eine vollständige Eigenentwicklung}
Segelboote im Grössenbereich von 2 m bis 2.5 m werden weder kommerziell angeboten noch industriell gefertigt. Für das vorliegende Projekt kann daher nicht auf einen bestehenden Bootstypus zurückgegriffen werden. Ein einfacher Erwerb eines solchen Bootes, um es in ein autonomes Boot umzubauen, würde der Zielsetzung widersprechen, selbst wenn ein entsprechendes Exemplar käuflich erworben werden könnte.

Für Modellbauer werden Bausätze für ferngesteuerte Segelboote angeboten, allerdings nur bis zu einer Gesamtlänge von etwa 1,6 m. Die Bootskörper dieser Modelle sind zu knapp bemessen, um die Navigations- und Steuerungselekronik und die Energieversorgung unterzubringen. Sodann werden Bausätze für bemannte kleine Segelboote wie Optimisten etc. angeboten. Diese kosten mehrere tausend Franken und überschreiten die Maximallänge von 2,5 m. Schliesslich werden fertige Baupläne für Modellsegelboote zum Kauf angeboten. Diese Modelle erreichen mit ganz wenigen Ausnahmen nicht die erforderlichen Dimensionen und sind zu Lasten der Stabilität auf optimale Segeleigenschaften ausgelegt. 

Da nicht auf ein bestehendes Modell zurückgegriffen werden kann, muss das Segelboot für dieses Projekt auf jeden Fall vollständig neu entwickelt werden. 



