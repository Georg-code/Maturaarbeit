
\chapter{Konzeptentwicklung }
\label{chap:konzeptentwicklung}
\section{Zielsetzung}
Das Ziel dieser Arbeit ist es, ein autonomes Segelboot zu entwerfen, zu konstruieren und zu bauen. Dabei soll günstiges und einfach zu verarbeitendes Konstruktonsmaterial verwendet werden, das in Baumärkten erworben werden kann, mit Werkzeugen gearbeitet werden, die in einer gut bestückten Handwerkerwerkstatt vorhanden sind, die Navigation und Steuerung selbst entworfen werden und die dafür notwendigen Programme selbst entwickelt werden. 

Dazu soll mittels eines CAD-Programmes (CAD = Computer Aided Design) computergestützt ein Segelboot entworfen werden, um anschliessend nach den mit dem CAD-Programm erstellten Plänen gebaut zu werden. Das Segelboot soll mit allen zur autonomen Navigation und Steuerung erforderlichen Geräten, Rechnern und Sensoren sowie der zu deren Betrieb erforderlichen autonomen Energieversorgung ausgerüstet werden. Schliesslich soll das Segelboot mit Programmen ausgestattet werden, die es ihm erlauben, bei allen möglichen Windrichtungen den vordefinierten Zielpunkt anzusteuern und selbst bei einer Veränderung der Windrichtung selbständig zu erreichen.

\section{Methodische Vorgehen}

In einem ersten Schritt wird ein Überblick über die bekannten autonomen Segelboote und Segelbootprojekte gewonnen. Danach werden die Anforderungern detailliert analysiert. 
In einem ersten Teil wird die 

Konzeption, der Entwurf, die Konstruktion und der Bau des Bootskörpers sowie des Segels beschrieben. In einem zweiten Teil wird die

  Im Abschnitt über die technische/elektrische Ausrüstung des Bootes, insbesondere der verwendet Rechner und die Überlegungen zu dessen Wahl, die eingesetzten Sensoren, die Stromversorgungslösung und die Lengsteuerung diskutiert.

 In einem nächsten Teil werden nach einer knappen Einführung in die Grundlagen der Physik des Segelns, die Überlegungen zur Architektur der Naviationssoftware, die Navigation und die Wegfindung diskutiert

 In einem ….. Teil wird der Bau der Bestandteile des Bootskörpers und des Segels und anschliessende Zusammenbau und die Montage beschrieben 

\section{Anforderungsanalyse}
Ausgehend von der Zielsetzung werden in einem ersten Schritt die Anforderungen an das autonome Segelboot analysiert und die Kriterien festgehalten, Das Boot muss über
\begin{itemize}
    \item stabil schwimmen
    \item den Wind nutzen
    \item günstig sein
    \item steuerbar sein
    \item an Land mit normalen Motorfahrzeugen an ein Gewässer transportiert werden können
    \item zwischen 2 und 2,5 m lang sein
    \item selbständig den Weg ins Ziel finden und ansteuern können
    \item selbständig mehrere, vorgegebene Ziel in der vorgegebenen Reihenfolge ansteuern können
    \item über eine autarke Energieversorgung verfügen und über eine Energiereserve von 72 h verfügen 
    \item muss auf Schweizer Gewässern betrieben werden können
    \item zwischen 2 und 2,5 m lang sein
    \item Segelfläche von weniger als 15m2
    \item Zieleingabe kann drahtlos erfolgen (Wlan)
    \item     darf eine Länge von 2.5 m nicht überschreiten (wegen Zulassung)
    \item Muss über einen Namen verfügen und mit dem Namen und der Adresse des Eigentümers versehen sein 
    \item bei starkem Wellengang und starkem Wind nicht kentern
    \item sich in einem See bewegen können, ohne auf Grund zu laufen, wenn dem Boot eine Karte des Sees eingespeist wird ????
\end{itemize}

Kein Fokus auf Geschwindigkeit
Kein Fokus auf der Ermittlung des schnellsten Weges, um alle vorgegebenen Zielpunkte abzufahren (Problem des Handlungsreisenden)
Keine Erkennung von Hindernissen wie andere Boote etc.
 

\section{Entscheid für eine vollständige Eigenentwicklung}
Segelboote im Grössenbereich von 2 m bis 2.5 m werden weder kommerziell angeboten noch industriell gefertigt. Für das vorliegende Projekt kann daher nicht auf einen bestehenden Bootstypus zurückgegriffen werden. Ein einfacher Erwerb eines solchen Bootes, um es in ein autonomes Boot umzubauen, würde der Zielsetzung widersprechen, selbst wenn ein entsprechendes Exemplar käuflich erworben werden könnte.

Für Modellbauer werden Bausätze für ferngesteuerte Segelboote angeboten, allerdings nur bis zu einer Gesamtlänge von etwa 1,6 m. Sodann werden Bausätze für bemannte Segelboote wie Optimisten etc. angeboten. Diese kosten aber mehrere tausend Franken und sind ausnahmslos mit Textilsegeln ausgerüstet.  
Schliesslich werden fertige Baupläne für Modellsegelboote angeboten. Sämtliche angebotenen Pläne sehen Textilsegel vor. Ausserdem erreichen mit ganz wenigen Ausnahmen nicht die erforderlichen Dimensionen. 

Da nicht auf ein vorhandenes Modell zurückgegriffen werden kann, muss das Segelboot für dieses Projekt daher vollständig neu entwickelt werden. 



\section{Gesetzliche Rahmenbedingungen}

Das projektierte autonome Segelboot ist ein Segelschiff im Sinne der Binnenschifffartsverordnung (siehe Kapital 1.2 oben) und fällt damit grundsätzlich unter die einschlägigen gesetzlichen Zulassungs-, Betriebs- und Verkehrsbestimmungen. Da es aber kürzer als 2.5 m ist, ist es von der Kennzeichnungspflicht mit dem behördlich zugeteilten Kennzeichen ausgenommen (Artikel 16 Abs. 2 Buchstabe b der Binnenschifffartsverordnung). Es benötigt damit gemäss Art. 92 der Binnenschiffartsverordnung auch keinen Schiffsausweis, womit auch eine behördliche Zulassungsprüfung entfällt. Das Boot muss aber einen Schiffsnamen tragen, der sich aus Buchstaben und Zahlen zusammensetzen kann, sowie mit dem Namen und der Adresse des Eigentümers versehen sein (Art. 16 Abs. 3 Binnenschifffartsverordnung). Für die Führung des autonomen Segelbootes ist kein Schiffsführerausweis erforderlich, solange die Segelfläche nicht mehr als 15 Quadratmeter beträgt. Das vorgesehen Segel erreicht diese Grösse bei weitem nicht. 

Beim Betrieb muss das autonome Segelboot 

\href{https://www.fedlex.admin.ch/eli/cc/1979/337_337_337/de\#art_42}{\textbf{Art. 42}}\textsuperscript{\href{https://www.fedlex.admin.ch/eli/cc/1979/337_337_337/de\#fn-d6e2948}{103}}\href{https://www.fedlex.admin.ch/eli/cc/1979/337_337_337/de\#art_42}{ Besondere Regeln}Schiffe, die kürzer sind als 2,50 m (Art. 16 Abs. 2 Bst. b), Strandboote und dergleichen (Art. 16 Abs. 2 Bst. c) dürfen nur in der inneren Uferzone (150 m) oder im Abstand von höchstens 150 m um sie begleitende Schiffe herum verkehren.

Licht Art. 25 Abs. 2 
\textsuperscript{2} Segelschiffe, die nur unter Segel fahren, führen bei Nacht und bei unsichtigem Wetter:

a. getrennte Seitenlichter und ein Hecklicht; b. ein Kombinations-Seitenlicht und ein Hecklicht; c ein Dreifarben-Topplicht; oderd. ein weisses Rundumlicht. 


\href{https://www.fedlex.admin.ch/eli/cc/1976/725_724_724/de\#art_16}{\textbf{Art. 16}  Gesetz 
Schiffsführung}\textsuperscript{1} Jedes Schiff muss einen verantwortlichen Führer haben.

\textsuperscript{2} Schiffsführer ist, wer die tatsächliche Befehlsgewalt innehat.

 

Schalleneichengabe. Ab dem begleitenden Boot. Art. 33 gemäss den an dieses gestetllten Anforderungen.

 Das autonome Segelboot darf auf Schweizer Gewässern nicht ohne "Aufsicht" autonom operieren dürfen. 






