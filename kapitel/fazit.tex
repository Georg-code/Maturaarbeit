\chapter{Verbesserungspotenzial und Ausblick}
\label{chap:fazit}


\section{Verbesserungspotenzial}
\subsection{Rumpf}
Bei einer Wiederholung des Bootsbaus würde die Konstruktion und der Bau des Rumpfes anders erfolgen. Das Verwenden von 3D- gedruckten Elementen hat sich beim Spitz als sehr erfolgreich herausgestellt. Daher würde bei einem zweiten Bau auf die Balsaholzbeplankung verzichtet und stattdessen möglichst lange und breite 3D-gedruckte Elemente verwendet werden. Diese würden mit der CAD Software entworfen und könnten dann in der idealen Form gedruckt werden. Damit würden Verspannungen im Holz und die daraus resultierenden Ungenauigkeiten vermieden, die im daran anschliessenden Arbeitsschritt der Beschichtung mit glasfaserverstärktem Epoxidharz sehr viel Arbeit verursacht hat und sehr zeitaufwendig war.

Sodann würden Spanten von geringerer Stärke aber in grösserer Zahl vorgesehen. Diese dünneren Spanten könnten dann sehr einfach mit dem Lasercutter erstellt werden.


\subsection{Ruder}
Bei einer Wiederholung des Bootsbaus würde das Ruder nicht am Ende des Bootskörpers vorgesehen, sondern vollständig unter Wasser am Heck des Bootskörpers befestigt und über eine Welle bewegt werden, die den Bootsboden vertikal schneidet. Damit könnten für die Rudersteuerung ein einfacher und günstiger Servomotor im Schiffskörper montiert und betrieben werden. Der zuverlässigen Abdichtung der Wellendurchführung müsste dabei grosse Aufmerksamkeit geschenkt werden. Ausserdem würde das Ruder im 3D-Druckverfahren hergestellt. 
\section{Anwedungsbereiche Autonomer Segelboote}

Autonome Segelboote werden nie dieselbe Aufmerksamkeit wie autonome Landfahrzeuge auf sich ziehen. Trotzdem gibt es einige Anwendungsfälle, wo sie ihren alternativen überlegen sind.

\subsection{Bedeutung autonomer Segelboote für die Erforschung des Unterwasserlebens}

Autonome Segelboote bieten eine vielversprechende technologische Grundlage für die Erforschung mariner Ökosysteme. Aufgrund ihrer Fähigkeit, sich energieeffizient, langfristig und über grosse Distanzen fortzubewegen, eignen sie sich insbesondere für die kontinuierliche Erhebung von Umweltdaten in schwer zugänglichen oder abgelegenen Meeresregionen. Im Vergleich zu bemannten Forschungsschiffen oder motorisierten unbemannten Systemen zeichnen sich autonome Segelplattformen durch ihren weitgehend emissionsfreien Betrieb aus, da sie weder fossile Brennstoffe benötigen noch durch den Einsatz von Solarenergie nennenswerte Emissionen verursachen.

Diese Charakteristika prädestinieren autonome Segelboote für den Einsatz in der Langzeitüberwachung mariner Lebensräume. Anwendungsfelder umfassen unter anderem die Verfolgung von Wanderbewegungen mariner Organismen, die Erfassung physikochemischer Wasserparameter wie Temperatur, Salzgehalt oder pH-Wert sowie die Untersuchung klimabedingter Veränderungen in ozeanischen Habitaten. Die vergleichsweise niedrigen Betriebskosten und die Möglichkeit zur Skalierung des Einsatzes ermöglichen eine signifikante Erweiterung der räumlichen und zeitlichen Abdeckung ökologischer Messkampagnen.

Ein weiterer Vorteil liegt in der geringen akustischen Störwirkung dieser Plattformen. Aufgrund ihres nahezu lautlosen Antriebs durch Windkraft beeinträchtigen autonome Segelboote die akustische Umwelt nur minimal, was sie zu einem geeigneten Instrument für bioakustische Studien macht. So können beispielsweise bislang schwer erfassbare Kommunikationsformen mariner Säugetiere, etwa von Walen, detaillierter untersucht werden. Dies eröffnet neue Perspektiven für das passive akustische Monitoring und die nicht-invasive Erforschung der Meeresfauna.

Darüber hinaus zeichnen sich autonome Segelboote durch eine hohe operative Flexibilität aus. Im Gegensatz zu stationären Messsystemen wie Bojen oder festen Plattformen können sie aktiv auf Umweltveränderungen reagieren und gezielt in ökologisch relevante Gebiete navigieren. Diese Mobilität erlaubt es, dynamische Prozesse wie Planktonblüten, Strömungsänderungen oder Tierwanderungen in Echtzeit zu verfolgen und mit hoher räumlich-zeitlicher Auflösung zu untersuchen. Insgesamt ermöglichen autonome Segelboote eine adaptive, präzise und ressourcenschonende Datenerhebung, die mit konventionellen Methoden nur eingeschränkt realisierbar wäre.



\subsection{Kartografierung}




\subsection{Verteidigung und Überwachung}




