

\chapter{Fazit}
\label{chap:fazit}


\section{Bewertung des autonomen Segelschiffs}




\section{Diskussion des Verbesserungspotenzials}
\subsection{Rumpf}
Bei einer Wiederholung des Bootsbaus würde die Konstruktion und der Bau des Rumpfes anders erfolgen. Die Verwenden von 3D- gedruckten Elementen hat sich beim Spitz als sehr erfolgreich herausgestellt. Daher würde bei einem zweiten Bau auf die Balsaholzbeplankung verzichtet werden und dafür möglichst lange und breite 3D-gedruckte Elemente verwendet. Diese würden mit der CAD Software entworfen und in der idealen Form gedruckt. Damit  würden Verspannungen im Holz und die daraus resultierenden Ungenauigkeiten  vermieden werden, die im daran anschliessenden Arbeitsschritt, der Beschichtung mit glasfaserverstärktem Epoxiharz sehr viel und zeitaufwendige Arbeit verursacht hat.

Sodann würden Spanten von geringerer Stärke, dafür in grösserer Zahl vorgesehen. Diese Spanten könnten dann mit dem Lasercutter erstellt werden.  
\subsection{Elektronik}
Bei einer Wiederholung des Bootsbaus sollte für die Verkabelung der elektronischen Bauteile eine Platine individuell entworfen und bei einem Auftragsfertiger in Auftrag gegeben werden. Der Aufbau und die Fehlersuche würde damit enorm erleichtert. Auch könnte der Aufbau damit sehr viel ordentlicher vorgenommen werden.

Sodann sollte ein wasserdichtes Gehäuse entworfen und gebaut werden, in dem der Raspberry Pi Zero W 1.1, die Platine, und der Pegelumsetzer Platz finden. Auch für den Energiespeicher und die Ladeelektronik sollte ein wasserdichtes Gehäuse entworfen und gebaut werden. Damit könnte die Gefahr eines Kurzschlusses bei einem Wassereinbruch, z.B. bei hohem Wellengang, stark reduziert werden.
\subsection{Segel}
Bei einer Wiederholung des Bootsbaus sollten im Mast zwei Löcher zur Verankerung der Carbonrohre vorgesehen werden, an welchen das Sailflap montiert wird. 

Die Oberfläche des Grosssegels und des Sailflaps würde mit einer dünnen Schicht glasfaserverstärktem Epoxidharz geschützt. 
\subsection{Kiel}
Bei einer Wiederholung des Bootsbaus sollte der Kiel aus Massivholzbrettern statt verleimten Platten aufgebaut werden, um die Gefahr eines Bruchs bei übermässigem Seitendruck oder einer Grundberührung weiter zu vermindern. Ausserdem sollte der Kiel mit einer dicken Schutzschicht aus glasfaserverstärktem Epoxidharz eingefasst und geschützt werden. 
\subsection{Ruder}
Bei einer Wiederholung des Bootsbaus würde das Ruder nicht am Ende des Bootskörpers vorgesehen, sondern vollständig unter Wasser am Heck des Bootskörpers befestigt und über eine Welle bewegt werden, die den Bootsboden verikal schneidet. Damit könnten für die Rudersteuerung ein einfacher und günstiger  Servomotor im Schiffskörper montiert und betrieben werden. Der zuverlässigen Abdichtung der Wellendurchführung müsste dabei grosse Aufmerksamkeit geschenkt werden. Ausserdem würde das Ruder im 3D-Druckverfahren hergestellt. 
\section{Ausblick auf zukünftige Entwicklungen im Bereich der autonomen Segelboote}
Autonome Segelboote werden nie dieselbe Aufmerksamkeit wie autonome Landfahrzeuge auf sich ziehen. 

Ihre Einsatzmöglichkeiten sind begrenzt. Dies finden sich vornehmlich in Nischen. Da Windstärke und Richtung auf absehbare Zeit nicht zuverlässig prognostiziert werden können, sind weder Reiserouten noch Reisezeiten autonomer Segelboote zuverlässig planbar oder auch nur annäherungsweise prognostizierbar. DAs setzt einem kommerziellen Einsatz enge Grenzen. Eine weite Verbreitung autonomer Segelboote darf daher nicht erwartet werden. 

Autonome Segelboote könnten aber beispielsweise im Monitoring von Gewässern ein wichtiges Instrument werden, um grosse Räume über längere Zeiträume kostengünstig zu überwachen. Dabei ist zwischen dem maritimen Einsatz und dem Einsatz auf Binnengewässern zu unterscheiden. 

Für den maritimen Einsatz werden sich deutlich grössere Boote als das in diesem Projekt erstellte Boot durchsetzen. solche eigenen sich ideal für Monitorinaufgaben in abgelegenen Meeresgegenden wie dem Südpolarmeer oder Teilen des indischen und pazifischen Ozeans., da sie ohne weiteres mehrere Monate oder gar wenige Jahre im autonomen Einsatz stehen können. Damit könnten beispielsweise Erkenntnisse über die Entwicklung von Fischpopulationen, die Wasserqualität oder Wassertemperatur, die Vermüllung aber auch Wetterdaten gewonnen werden. 

Der Einsatz von autonomen Booten der Grössenordnung wie in der  vorliegenden Arbeit auf dengrossen Schweizer Seen ist wenig wahrscheinlich, da diese von einer Vielzahl von Freizeitbooten befahren werden, die eine Gefahr für das autonome Segelboot darstellen. Viel eher ist an einen Einsatz solcher Boote auf den zahllosen Seen in den nordischen Ländern zu denken, die in sehr dünn besiedelten Gegenden liegen, und bei denen kaum eine Infrastruktur vorhanden ist, die eine gleichwertige Überwachung auf andere Weise erlauben würde.