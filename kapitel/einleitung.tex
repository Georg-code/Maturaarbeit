
\chapter{Einleitung }
\label{chap:einleitung}
Segelboote waren vor der Erfindung motorisierter Boote und Schiffe der Schlu¨ssel zur Globalisierung. Dank der Kraft des Windes konnten Menschen damals riesige Distanzen überwinden und Güter selbst über Ozeane transportieren. Seit der Erfindung der der Dampfmaschine und der damit ausgelösten industriellen Revolution dienen sie jedoch fast nur noch als Vergnu¨gungs- oder Sportgeräte. Mit der immer verfu¨ gbar werde- derderen Technologie des autonomen Fahren können Segelboote einen erneuten Aufschwung erleben.


\section{Heutige Verbreitung von Autonomen Segelbooten}
Während Berichte über Entwicklungen und Fortschritte autonomer Strassenfahrzeuge fast täglich in der Presse erscheinen und sich die bedeutendsten und kapitalkräftigsten Unternehmen der Fahrzeugindustrie und Informatik in einem harten Wettbewerb um die Führerschaft bei deren Entwicklung befinden (siehe https://de.wikipedia.org/wiki/Autonomes\_Fahren), verläuft die Entwicklung von autonome Segelboten ………Nische 

\section{Was sind autonome Segelboote}
\subsection{Segelboot}
Ein Schiff ist in Wasserfahrzeug oder ein anderer zur Fortbewegung auf oder unter der Wasseroberfläche bestimmter Schwimmkörper, oder ein schwimmendes Gerät (Artikel 2 Abs. 1 lit. a Ziffer 1 der Verordnung über die Schifffahrt auf schweizerischen Gewässern (Binnenschifffahrtsverordnung, BSV) vom 8. November 1978. Ein Boot ist damit ein Schiff, wobei der Begriff Schiff als Überbegriff für Wasserfahrzeuge gilt und der Begriff Boot regelmässig zur Bezeichnung kleineren Wasserfahzeugen wie Ruderbooten, Sportboote, Paddelbooten etc. dient, die in der Regel nicht eingedeckt sind (https://de.wikipedia.org/wiki/Boot). 

Ein Segelschiff ist ein Schiff, das für die Fortbewegung mit Segeln versehen ist; ein Segelschiff, das mit oder ohne gesetzte Segel unter Motor fährt, gilt im Sinne der Verkehrsvorschriften als Schiff mit Maschinenantrieb (Artikel 2 Abs. 1 lit. a Ziffer 9 der Verordnung über die Schifffahrt auf schweizerischen Gewässern (Binnenschifffahrtsverordnung, BSV) vom 8. November 1978. Ein Schiff ist folglich nur dann ein Segelschiff, wenn es 

 (a) für die Fortbewegung mit Segeln versehen ist, und 

 (b) über keinen Motor für die Fortbewegung verfügt oder einen vorhandenen Motor nicht dafür einsetzt. 

Der Einsatz von Motoren für andere Zwecke als der Fortbewegung, zum Beispiel für das Setzen von Segeln oder das Bewegen des Ruders, ändert aber nichts an der Klassierung eines Wasserfahrzeuges als Segelschiff. 

\subsection{Autonomie}
Unbemannt und ohne eigene Mannschaft und nicht ferngesteuert (Gegensatz zum Roboter)

Autonom ist ein Segelboot, das selbst, also autonom und ohne eigene Mannschaft operieren kann. Es benötigt einzig die Vorgabe ein es Ziels und findet danach selbständig den Weg an dieses vorgegebene oder allenfalls mehrere vorgegebene Ziele. Autonome Boote unterscheiden sich damit von Roboterschiffen oder Drohenschiffen, die ferngesteuert operiert werden. 

Ein autonomes Segelboot muss nach einer einmaligen Vorgabe eines Zielpunktes, seine Position auf dem Gewässer selbständig ermitteln, das Ziel unter Nutzung der Windkraft selbständig ansteuern, den gewählten Kurs den herrschenden und sich ändernden Umweltbedingungen wie Wind oder Strömung selbständig so anpassen, dass es das Ziel ohne Eingriffe einer Mannschaft unter Nutzung der Windenergie planmässig und nicht nur zufällig erreicht.


\section{Weshalb Autonome Segelboote an Bedeutung gewinnen könnten}

\section{Mögliche Konkrete Einsatzbereiche}
\subsection{Schadstoffbelastung am Chemieareal Uetikon am See}
Das Chemieareal ist ein Bereich direkt am Zürichsee, welches durch jahrelange Umweltverschmutzung und Belastung von Schwermetallen geprägt wurde. Beim Umbau des Areals in eine Schule und Wohnungskomplexe müssen die toxischen Überresten der Industrie entfernt werden. Dieser Entfernungsprozess gilt als sicher und viele Massnahmen um eine Vermeindung von Freisetzung von toxischem Material werden getroffen. Jedoch werden Gewässerproben um die Einhaltung der Wasserqualität nur sporadisch gemacht. Es Gefährt, welches während den Arbeiten einen grossen Bereich sehr kostengünstig überwachen könnte, wäre hier von Vorteil. Da an einem Segelboot Sensoren jegliche Art angebracht werden können, kann neben Verunreinigungen auch durch SONAR der Seeboden kartografiert werden oder das verhalten von Fischpopulationen überwacht werden. All dies sind äusserst wertvolle Daten, welche bis jetzt nur schwer regelmässig zu beschaffen waren. Sobald die Arbeiten abschlossen sind, könnte der Bereich für einen längeren Zeitraum überwach. um zu überprüfen ob die gemachte Arbeit auch den Annahmen standhält.

\subsection{Grosse Seen}
// Grosse seen komplett monitoren und über das ganze Jahr veränderungen am Seeboden feststellen

\section{Zielsetzung}
Das Ziel dieser Arbeit ist ein autonomes Segelboot von Grund auf neu zu entwickeln. 

Dazu soll mittels eines \href{https://www.overleaf.com/project/64c425dc65e5d82bb7de603c\#_bookmark0}{CAD}-Programmes (CAD = Computer Aided Design) computergestützt ein Segelboot entworfen werden, um anschliessend nach den mit dem CAD-Programm erstellten Plänen gebaut zu werden. Das Segelboot soll mit allen zur autonomen Navigation und Steuerung erforderlichen Geräten, Rechnern und Sensoren sowie der zu deren Betrieb erforderlichen autonomen Energieversorgung ausgerüstet werden. Schliesslich soll das Segelboot mit Programmen ausgestattet werden, die es ihm erlauben, bei allen möglichen Windrichtungen den vordefinierten Zielpunkt anzusteuern und selbst bei einer Veränderung der Windrichtung selbständig zu erreichen.

Kostengünstig mit Materialen aus Baumarkt und Elektronikversand und Heimwerkerausrüstung


\section{Aufbau der Arbeit}
In einem ersten Teil wird die 

 Konzeption, der Entwurf, die Konstruktion und der Bau des Bootskörpers sowie des Segels beschrieben. In einem zweiten Teil wird die

  Im Abschnitt über die technische/elektrische Ausrüstung des Bootes, insbesondere der verwendet Rechner und die Überlegungen zu dessen Wahl, die eingesetzten Sensoren, die Stromversorgungslösung und die Lengsteuerung diskutiert.

 In einem nächsten Teil werden nach einer knappen Einführung in die Grundlagen der Physik des Segelns, die Überlegungen zur Architektur der Naviationssoftware, die Navigation und die Wegfindung diskutiert

 In einem ….. Teil wird der Bau der Bestandteile des Bootskörpers und des Segels und anschliessende Zusammenbau und die Montage beschrieben 

  