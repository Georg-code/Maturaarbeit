
\chapter{Einleitung }
\label{chap:einleitung}

\section{Motivation}
Die Beobachtung von Wasserqualität wird in der Schweiz Hauptsächlich vom eawag durchgeführt. Das ETH Forschungsinstitut beschäftig sich u. a. auch mit der Wasserqualität von Schweizer Seen. Die Datenerhebung ist jedoch entweder sehr Arbeitsintensiv oder sie funktioniert nur Sationär. Ein System welches, Wasserproben aunehmen kann existiert in der Schweiz so noch nicht.
\\
Das Ziel ist es ein gefährt zu konstruieren und zu bauen, welches möglichst viele Einsatzmöglichkeiten auf Schweizern Gewässern hat. Es sollte lang 

\section{Zielsetzung}



\begin{itemize}
    \item Autonom Navigieren
    \item Energieeffizient in der Operation
    \item Wartungsarm
    \item Möglichst Kostengünstig
\end{itemize}

Wenn man sich diese Kriterien anschaut, ist der Blick weg von Autonomen Segelbooten nicht mehr weit. Dies wird jedoch weiter hinten genauer erläutert

\section{Überblick über den Aufbau der Arbeit}
(Am Ende schreiben)