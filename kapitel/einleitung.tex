
\chapter{Einleitung }
\label{chap:einleitung}
Segelboote waren vor der Erfindung der motorisierten Booten der Schlüssel zur Globalisierung. Dank der Kraft des Windes konnten Menschen und Güter über damals unerdenkliche Distanzen transportiert werden. Seit der industriellen Revolution und der Dampfmaschine dienen sie jedoch nur noch als Vergnügungs- oder Sportgefährt. Mit der immer verfügbar werdederderen Technologie des autonomen Fahren können diese einen erneuten Aufschwung erleben. 

\section{Mögliche Konkrete Einsatzbereiche}
\subsection{Schadstoffbelastung am Chemieareal Uetikon am See}
Das Chemieareal ist ein Bereich direkt am Zürichsee, welches durch jahrelange Umweltverschmutzung und Belastung von Schwermetallen geprägt wurde. Beim Umbau des Areals in eine Schule und Wohnungskomplexe müssen die toxischen Überresten der Industrie entfernt werden. Dieser Entfernungsprozess gilt als sicher und viele Massnahmen um eine Vermeindung von Freisetzung von toxischem Material werden getroffen. Jedoch werden Gewässerproben um die Einhaltung der Wasserqualität nur sporadisch gemacht. Es Gefährt, welches während den Arbeiten einen grossen Bereich sehr kostengünstig überwachen könnte, wäre hier von Vorteil. Da an einem Segelboot Sensoren jegliche Art angebracht werden können, kann neben Verunreinigungen auch durch SONAR der Seeboden kartografiert werden oder das verhalten von Fischpopulationen überwacht werden. All dies sind äusserst wertvolle Daten, welche bis jetzt nur schwer regelmässig zu beschaffen waren. Sobald die Arbeiten abschlossen sind, könnte der Bereich für einen längeren Zeitraum überwach. um zu überprüfen ob die gemachte Arbeit auch den Annahmen standhält.

\subsection{Grosse Seen}
// Grosse seen komplett monitoren und über das ganze Jahr veränderungen am Seeboden feststellen

\section{Zielsetzung}
Das Ziel dieser Arbeit ist ein Segelboot von Grund auf zu Entwickeln, es vorher Computergestützt mittels \ac{cad} zu zeichnen und anschliessend zu bauen. \\
Ebenfalls soll das Boot alle grundlegenden Funktionen zu autonomen Navigation enthalten. Schlussendlich sollte das Segelboot in der Lage sein unter verschiedenen Windstellungen Vordefinierten Zielpunkt zu erreichen können. Ebenfalls soll mit dem Abschluss der Arbeit ein Segelboot speziell für einen Autonomen Einsatzbereich entwickelt werden. 

