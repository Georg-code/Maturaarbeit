
\chapter{Einleitung }
\label{chap:einleitung}

\section{Motivation}
Die Beobachtung von Wasserqualität, Fischbestand und Temperatur wird in der Schweiz nur sehr sporadisch durchgeführt. Operationen welche bereits existiern involvieren meist Menschen und Motorisierte Boote. Dies bedeutet vor allem eins: Viel Aufwand, was im endefekt auch viel Geld bedeutet.. \\
Ein Schiff ähnliches Gefährt, welches selbstständig nicht Zeitkritische werte erfassen könnte, wäre hier von Vorteil.
\\
Man könnte, im Gegensatz zu bereits existierenden Boyen (pls cite), auch Daten über grössere Gewässer Sammeln ohne eine Vielzahl davon Platzieren zu müssen. 
\\

\section{Zielsetzung}
Um dies in die Tat umzusetzen, muss ein Gefährt entwickelt werden, welches lange und ohne menschlichen Einfluss autonom navigieren kann. 

\begin{itemize}
    \item Autonom Navigieren
    \item Energieeffizient in der Operation
    \item Wartungsarm
    \item Möglichst Kostengünstig
\end{itemize}

Wenn man sich diese Kriterien anschaut, ist der Blick weg von Autonomen Segelbooten nicht mehr weit. Dies wird jedoch weiter hinten genauer erläutert

\section{Überblick über den Aufbau der Arbeit}
(what?)