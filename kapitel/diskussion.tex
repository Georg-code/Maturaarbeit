

\chapter{Ergebnisse und Diskussion }
\label{chap:diskussion}

\section{Bewertung des autonomen Segelschiffs}

\section{Diskussion von Herausforderungen und Verbesserungspotenzial}
\subsection{Rumpf}
Bei einer Wiederholung des Bootsbaus würde die Konstruktion und der Bau des Rumpfes anders erfolgen. Die Verwenden von 3D- gedruckten Elementen hat sich beim Spitz als sehr erfolgreich herausgestellt. Daher würde bei einem zweiten Bau auf die Balsaholzbeplankung verzichtet werden und dafür möglichst lange und breite 3D-gedruckte Elemente verwendet. Diese würden mit der CAD Software entworfen und in der idealen Form gedruckt. Damit  würden Verspannungen im Holz und die daraus resultierenden Ungenauigkeiten  vermieden werden, die im daran anschliessenden Arbeitsschritt, der Beschichtung mit glasfaserverstärktem Epoxiharz sehr viel und zeitaufwendige Arbeit verursacht hat.

Sodann würden Spanten von geringerer Stärke, dafür in grösserer Zahl vorgesehen. Diese Spanten könnten dann mit dem Lasercutter erstellt werden.  

\section{Ausblick auf zukünftige Entwicklungen}
Autonome Segelboote werden nie dieselbe Aufmerksamkeit wie autonome Landfahrzeuge auf sich ziehen. 

Ihre Einsatzmöglichkeiten sind begrenzt und belegen vornehmlich Nischen. Da der Windstärke und Richtung auf absehbare Zeit nicht zuverlässig prognostiziert werden können, sind weder Routen noch Reisezeiten autonomer Segelboote zuverlässig planbar oder auch nur annäherungsweise prognostizierbar. Sie könnten aber beispielsweise im Monitoring von Gewässern ein wichtiges Instrument werden, um grosse Räume über  längere Zeiträume kostenkünstig zu überwachen. Damit könnten beispielsweise Erkenntnisse über die Entwicklung von  Fischpopulationen, der Wasserqualität oder der Wassertempratur  gewinnen.

(Weltmeere einbringen? Langsame Logistik Klimaneutral?)
