

\chapter{Ergebnisse und Diskussion }
\label{chap:diskussion}

\section{Bewertung des autonomen Segelschiffs}

\section{Diskussion von Herausforderungen und Verbesserungspotenzial}
\subsection{Rumpf}
Bei einer erneuten Durchführung, würde der Bau des Rumpfes anders angegangen werden. Das verwenden von 3D gedruckten Elementen hat sich mit dem Spitz als sehr erfolgreich herausgestellt. Daher würde in einem zweiten Bau auf die Balsaholz beschichtung verzichtet werden und statdessen 3D gedruckte Elemente zum Einsatz kommen, welche dann schon der im CAD vorgesehenen Form entsprechen würden. Somit können Verspannungen im Holz vermieden werden, welche bei der Beschichtung mit Glasfasermatten zu sehr viel Arbeit führen-  

\section{Ausblick auf zukünftige Entwicklungen}
Autonome Segelboote werden nie einen gleichen hypestatus wie autonome Landfahrzeuge erreichen. Die Einsatzmöglichkeiten sind eher Begrenzt und die Routen nicht zuverlässig im voraus Planbar. Jedoch könnten sie vor allem im Monitoring von Gewässern ein wichtiges Instrument werden, um grosse Bereiche über einen längeren Zeitraum zu Überwachen und somit wichtige Erkentnisse über Fischpopulationen und Wasserqualität zu gewinnen.

(Weltmeere einbringen? Langsame Logistik Klimaneutral?)
