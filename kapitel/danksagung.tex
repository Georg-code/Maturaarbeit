

\chapter{Danksagung}
\label{chap:danksagung}
Das Projekt hätte ohne die wertvole Hilfe der vieler Personen nicht realisiert werden können. Ausdrücklich hervorheben möchte ich
\begin{itemize}
\item Thomas Zwick, mein Segellehrer, der mir als Fünjährigem meine ersten Segelstunden erteilt. Er hat mich nicht nur zum  Segeln gebracht, sondern hat mir für den Bau des Segelbootes in grosszügiger Weise seine Werkstatt zur Verfügung gestellt und mich zunächst in die Technik der Arbeit mit glasfaserverstärkten Kunststoffen eingeführt und danach bei diesen Arbeiten tatkräftig untersützt hat.
 \item Dr. Carola Ebenhoch, meine Betreuerin und ehemalige Physiklehrerin, welche mir bei dieser Arbeit sehr viel Freiraum eingeräumt, meine etwas waghalsige Idee ein komplettes Segelboot zu entwerfen, konstruieren und zu bauen gelassen hat.
  \item Raphael Barengo, ebenfalls Physiklehrer an der Kantonsschule Uetikon am See, der mir Zugang zu den 3D Druckern, dem Lasercutter und diveresen anderen Werkzeugen in der Makerhall der Schule ermöglicht hat.  
  \item Samuel Achermann, mein Mathematiklehrer an der Kantonsschule Uetikon am See, der sich in den Ferien Zeit genommen, mit mir Navigationsalgorithem durchzudenken und mir vor allem dabei geholfen hat, den Potential Fields Algorithmus zu verstehen.
  
  \item Gian-Andri Morf, mein Cousin und MSc ETH Masch. -Ing., von dem ich am Anfang des Projekts sehr viel zur Materialauswahl lernen konnte.
  \item Meine Mutter, welche mich während des ganzen Projektes geduldigst unzählige Male in die Werkstatt gefahren und von dort wieder abgeholt hat.
  \item  Mein Vater, den ich als Lektor missbrauchen durfte und der mich bei den rechtlichen Aspekten beraten hat.
   \item Samuel Hasenfratz, Mitarbeiter und Ingenieur bei Tribecraft AG, Zürich, der mir nicht nur das spannende Unternehmen vorgestellt, sondern sich auch viel Zeit genommen hat, einige Herausforderungen, vor allem bezüglich der Navigation die noch zu lösen waren, zu durchdenken oder gar eine Lösung zu finden. 
\item Distrelec Schweiz AG, Nänikon, die sich bereit erklärt haben, mein Projekt mit einer grosszügigen Sachspende in Form von elektronischen Bauteilen zu unterstützen. 
 
\end{itemize}
