

\chapter{Danksagung}
\label{chap:danksagung}
Das Projekt konnte nicht ohne die grosszügige Hilfe von einigen Leuten entstehen.
\begin{itemize}
\item Thomas Zwick, mein Segellehrer seit dem ich fünf bin, der mich zum einen zum Segeln gebracht hat, mir aber auch während des ganzen Zeitspanne des Projektes Raum in seiner Werkstatt gegeben hat und mich stark beim Beschichten des Bootes mit Glassfaser beraten, aber auch sehr stark geholfen hat.
 \item Dr. Carola Ebenhoch, meine Betreuerin und ehemalige Physiklehrerin, welche mir sehr viel Freiraum gegeben hat, meine etwas waghalsige Idee ein komplettes Segelboot zu entwerfen, konstruieren und zu bauen gelassen hat.
  \item Raphael Barengo, ebenfalls Physiklehrer an der Kantonsschule, der mir Zugang zu den 3D Druckern, dem Lasercutter und diveresen anderen Werkzeugen in der Makerhall gegeben hat.  
  \item Samuel Achermann, mein Mathematiklehrer hat sich in den Ferien Zeit genommen, mit mir Navigationsalgorithem durchzudenken und mir vorallem dabei geholfen hat den Potential Fields Algorithmus zu verstehen.
  
  \item Gian-Andri Morf, mein Cousin und Maschinebauer von dem ich vor allem am Anfang des Projekts sehr viel über Materialauswahl lernen konnte.
  \item Meine Mutter, welche mich während des Projektes sehr Regelmässig in die Werkstatt gefahren hat
  \item  Mein Vater, welcher die Arbeit Korrekturgelesen hat
   \item Samuel Hasenfratz, Mitarbeiter bei Tribecraft, der mir das spannende Unternehmen gezeigt hat, sich dann aber auch noch viel Zeit genommen hat, einige Herausforderungen, vor allem bezüglich der Navigation die noch zu lösen waren, zu durchdenken oder gar eine Lösung zu finden. 
\item Distrelec, Technikverstandhändler der sich dazu entschieden hat, mich bei meinem Projekt mit einigen Bauteilen zu unterstützen. 
 
\end{itemize}
