 \chapter{Danksagung}
\label{chap:danksagung}
Das Projekt hätte ohne die wertvolle Hilfe vieler Personen und Institutionen nicht realisiert werden können. Ausdrücklich hervorheben möchte ich:
\begin{itemize}
    \item \textbf{Thomas Zwick}, mein Segellehrer, der mir als Fünfjährigem meine ersten Segelstunden erteilt hat. Er hat mich nicht nur zum Segeln gebracht, sondern mir für den Bau des Segelbootes grosszügig seine Werkstatt zur Verfügung gestellt, mich in die Technik der Arbeit mit glasfaserverstärkten Kunststoffen eingeführt und mich bei diesen Arbeiten tatkräftig unterstützt.

    \item \textbf{Dr. Carola Ebenhoch}, meine Betreuerin und ehemalige Physiklehrerin, die mir bei dieser Arbeit sehr viel Freiraum eingeräumt und meine etwas waghalsige Idee, ein Segelboot von Grund auf zu entwerfen, konstruieren und zu bauen, mitgetragen hat.

    \item \textbf{Raphael Barengo}, Physiklehrer an der Kantonsschule Uetikon am See, der mir Zugang zu den 3D-Druckern, dem Lasercutter und diversen anderen Werkzeugen in der Makerhall ermöglicht hat.

    \item \textbf{Samuel Achermann}, mein Mathematiklehrer an der Kantonsschule Uetikon am See, der sich sogar in seinen Ferien Zeit genommen hat, mit mir Navigationsalgorithmen durchzudenken und mir vor allem dabei geholfen hat, den Potential-Fields-Algorithmus zu verstehen.

    \item \textbf{Gian-Andri Morf}, MSc ETH Maschineningenieur, mein Cousin, von dem ich am Anfang des Projekts viel über die Auswahl geeigneter Materialien lernen konnte.

    \item \textbf{Samuel Hasenfratz}, Ingenieur bei der Tribecraft AG in Zürich, der mir nicht nur das spannende Unternehmen vorgestellt, sondern sich auch viel Zeit genommen hat, um einige offene Herausforderungen – insbesondere zur Navigation – zu analysieren und gemeinsam Lösungsansätze zu entwickeln.

    \item \textbf{Meine Mutter}, die mich während des gesamten Projekts geduldig unzählige Male in die Werkstatt gefahren und wieder abgeholt hat.

    \item \textbf{Mein Vater}, den ich als Lektor „missbrauchen“ durfte und der mich in rechtlichen Fragen unterstützt hat.

    \item \textbf{Distrelec Schweiz AG}, Nänikon, die mein Projekt mit einer grosszügigen Sachspende in Form elektronischer Bauteile unterstützt haben.

    \item \textbf{Andreas Mantel}, mein ehemaliger Werklehrer und Eisenplastiker, der mir das Schweissen beigebracht und mich grosszügig beim Bau des Kiels unterstützt hat.

    \item \textbf{FabLab Luzern}, das mir das Segel CNC-gefräst und sogar bis nach Zürich geliefert hat.

    \item \textbf{PCBWay}, ein chinesisches Platinenfertigungsunternehmen, das mir freundlicherweise eine Iteration der Leiterplatte kostenlos zugesendet hat.

    \item \textbf{Andreas Claris}, Hausmeister der Kantonsschule Uetikon am See, der mich beim Bohren dicker Metallstücke unterstützt hat.

    \item \textbf{Stefanie Jörg}, Chemielehrperson an der Kantonsschule Uetikon am See, die mir die nötigen Chemikalien zur Reinigung der Platinen zur Verfügung gestellt hat.

    \item \textbf{Aaron Griesser}, Robotiklehrer an der Kantonsschule Uetikon am See und Elektrotechnikstudent, der mir das Entwickeln von Platinen beigebracht und mir während des gesamten Projekts bei elektronischen Fragen geholfen hat.

    \item \textbf{Carlos Niggli}, mein Bruder und Elektronikerlehrling am SLF in Davos, der mir beim Löten schwer zugänglicher Bauteile geholfen und mich mit fehlenden Widerständen versorgt hat.
\end{itemize}