

\chapter{Auflistung verwandter Arbeiten}
\label{chap:literaturübersicht}

In diesem Kapitel wird eine Übersicht über relevante Arbeiten präsentiert, die sich mit dem Bau und der Konstruktion oder der Autonomie von Segelbooten befassen.

Mit dem Bau und der Konstruktion befasst sich Schröder et. al. im confrence Paper \enquote{Development of a low-budget robotic sailboat}. Es wird ein EPS basierter Rumpfs beschrieben, sowie ein frei rotierendes Segels. Das Boot ist 1.2m lang und kostet nur ca. 350 Franken. Das Boot, sowie die Bordelektronik wird selbst entwickelt. \cite{10.1007/978-3-319-02276-5_2}
 
Ebenfalls ein autonomes Segelboot wird von Giger et. al. in \enquote{Design and Construction of the Autonomous Sailing Vessel AVALON} \ an der ETH Zürich konstruiert und gebaut. Das Ziel ist es, mit dem gebauten Katamaran autonom den Atlantik zu überqueren. \cite{giger_design_2009}

Mit dem Bau des Rumpfes eines Segelboots beschäftigt sich Ingrassia et. al. in \enquote{Parametric Hull Design with Rational Bézier Curves and Estimation of Performances}. Die digitale Konstruktion von Schiffskörpern  mithilfe von Bezière Kurvenwird  beschrieben. \cite{ingrassia_parametric_2021}

Ausschliesslich mit dem Segeldesign beschäftigt sich Tretow C. in \enquote{Design of a free-rotating wing sail for an autonomous sailboat}. Das Konzept eines Sailwings und dessen Funktionsweise wird beschrieben, sowie ein möglichst effizintes, aber auch robustes Segel entwickelt. \cite{Tretow2017DesignOA}

Ein high-level Ansatz zur Navigation mittels dem Q-Learning Ansatz verfolgen Silva Junior et. al. in \enquote{High-Level Path Planning for an Autonomous Sailboat Robot Using Q-Learning}. Der Maschine Learing Algorithmus basiert auf einem Rastermodell, in welchem sich das Segelboot bewegt. \cite{silva_junior_high-level_2020}

Ebenfalls mit der Navigation, jedoch einem deutlich einfacheren Modell, beschäftigen sich Jaulin und Le Bars in \enquote{{}A{} Simple Controller for Line Following of Sailboats}'. Ein einfacher, jedoch effizienter Ansatz zur Navigation über weite Distanzen wird beschrieben. Das Ziel ist es, das Segelboot einen vordefinierten Pfad möglichst genau entlang fahren zu lassen. \cite{sauze_simple_2013}

Einen ganz anderen Ansatz verfolgen Luc und Plumet et. al. in \enquote{Line following for an autonomous sailboat using potential fields method}, indem sie den Potential Fields Path Finding Algroithmus aufs Segeln angewenden. \cite{inproceedings}

Seit 2010 werden unter dem Namen MicroTransat Challenge regelmässig Wettbewerbe zur Überquerung des Atlantiks mit autonomen, vom Wind angetriebenen Booten organisiert, die nicht länger als 2,4 m sein dürfen. 
Damit sollen Entwickler auf der ganzen Welt motiviert werden, Boote zu entwickeln, welches in der Lage sind, den Atlantik zu überqueren. \cite{noauthor_microtransat_nodate}

Die kommerzielle Anwendung von autonomen Segelbooten wird vom US-Amerikanischen 
Startup-Unternehmen \enquote{Saildrone} vorangetrieben, welche eine Flotte an autonomen Segelbooten hauptsächlich für die Vermessung von Ozeanen einsetzt. \cite{noauthor_saildrone_nodate}


























