\begin{abstract}
Autonome Fahrzeuge gewinnen zunehmend an Bedeutung. Auch auf Gewässern bieten sie vielfältige Einsatzmöglichkeiten, darunter beispielsweise die Überwachung der Wasserqualität. Segelboote erweisen sich aufgrund ihrer Emissionsfreiheit und ihres geringen Energiebedarfs als besonders geeignet für solche Einsatzzwecke.

Bisher werden autonome Segelboote fast ausschliesslich für den maritimen Einsatz entwickelt. In dieser Arbeit wird die Entwicklung und der Bau eines vollständig autonomen Segelboots beschrieben, das für den Einsatz auf Binnengewässern vorgesehen ist. Das 2,2 Meter lange Boot hat einen Rumpf aus glasfaserverstärktem Kunststoff und ist mit einem Hartsegel mit Sailwing ausgestattet. Das selbstentwickelte einfache Navigations- und Steuerprogramm läuft auf einem leistungsschwachen ESP 32. Die Energieversorgung wird durch ein kleines Solarpanel in Kombination mit einen selbst gebauten Akkumulator sichergestellt.
\end{abstract}
